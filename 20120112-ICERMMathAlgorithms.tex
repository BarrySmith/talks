% \documentclass[handout]{beamer}
\documentclass{beamer}

\mode<presentation>
{
  \usetheme{default}
  \usefonttheme[onlymath]{serif}
  % \usetheme{Singapore}
  % \usetheme{Warsaw}
  % \usetheme{Malmoe}
  % \useinnertheme{circles}
  % \useoutertheme{infolines}
  % \useinnertheme{rounded}

  \setbeamercovered{transparent=5}
}

\usepackage[english]{babel}
\usepackage[latin1]{inputenc}
\usepackage{textpos,alltt,listings,multirow,ulem,siunitx}
\usepackage{pdfpages}
\newcommand\hmmax{0}
\newcommand\bmmax{0}
\usepackage{bm}

% font definitions, try \usepackage{ae} instead of the following
% three lines if you don't like this look
\usepackage{mathptmx}
\usepackage[scaled=.90]{helvet}
% \usepackage{courier}
\usepackage[T1]{fontenc}
\usepackage{tikz}
\usetikzlibrary[shapes,shapes.arrows,arrows,shapes.misc,fit,positioning]

% \usepackage{pgfpages}
% \pgfpagesuselayout{4 on 1}[a4paper,landscape,border shrink=5mm]

\usepackage{JedMacros}

\title{Mathematical Algorithms}
%\author{Jed Brown, Felipe Cruz, Matt Knepley, Richard Mills, John Owens, Jack Poulson, Ulrike Meier Yang}


% - Use the \inst command only if there are several affiliations.
% - Keep it simple, no one is interested in your street address.

\date{2012-01-12}

% This is only inserted into the PDF information catalog. Can be left
% out.
\subject{Talks}


% If you have a file called "university-logo-filename.xxx", where xxx
% is a graphic format that can be processed by latex or pdflatex,
% resp., then you can add a logo as follows:

% \pgfdeclareimage[height=0.5cm]{university-logo}{university-logo-filename}
% \logo{\pgfuseimage{university-logo}}



% Delete this, if you do not want the table of contents to pop up at
% the beginning of each subsection:
% \AtBeginSubsection[]
% {
% \begin{frame}<beamer>
%   \frametitle{Outline}
%   \tableofcontents[currentsection,currentsubsection]
% \end{frame}
% }

\AtBeginSection[]
{
  \begin{frame}<beamer>
    \frametitle{Outline}
    \tableofcontents[currentsection]
  \end{frame}
}

% If you wish to uncover everything in a step-wise fashion, uncomment
% the following command:

% \beamerdefaultoverlayspecification{<+->}

\begin{document}
\lstset{language=C}
\normalem

\begin{frame}
  \titlepage
\end{frame}

\begin{frame}{Unifying themes}
  \begin{itemize}
  \item We want to \alert{\large maximize science}.
    \begin{itemize}
    \item Kernel tuning and fusing helps sometimes
    \item Huge scope remains at problem formulation
    \end{itemize}
  \item Raise level of abstraction at which a problem is formally specified
    \begin{itemize}
    \item Improves reproducibility of scientific results
    \item Permits fusing of algorithmic stages, sometimes with better mathematical properties
      \begin{itemize}
      \item convergence proofs become more delicate
      \end{itemize}
    \end{itemize}
  \item Algorithmic optimality is crucial
  \end{itemize}
\end{frame}

\begin{frame}
  \begin{itemize}
  \item Multilevel methods
    \begin{itemize}
    \item matrix-free and nonlinear schemes
    \item reduce dependence on assembly and factorization
    \item smoothers natural to discretizations (e.g. multigrid with Neumann problems)
    \item coarse spaces for nonlocal problems (2nd kind, wave ray)
    \item smoothers for nonlocal problems
    \item reusing and fusing with different levels of analysis (MLMC, stability, sensitivity)
    \item traditional robustness and generality (anisotropy, heterogeneity, saddle points, null spaces)
    \end{itemize}
  \item (Nonlinear) elimination for arithmetic intensity/communication reduction
    \begin{itemize}
    \item at element level: Hybridizable DG
    \item at subdomain level: nonlinear Schwarz (ASPIN)
    \item at physics level: low-Mach, IMEX
    \end{itemize}
  \item Reconsidering high-coefficient/highly-parallel methods
    \begin{itemize}
    \item Strassen
    \item Alternative Fourier transform algorithms (FMM, butterfly, etc)
    \item Schulz iteration for inverting structured matrices
    \end{itemize}
  \end{itemize}
\end{frame}

\begin{frame}{Advanced Analysis}
  \begin{itemize}
  \item Uncertainty quantification
    \begin{itemize}
    \item intrusive vs unintrusive methods, multilevel
    \item uncertainty in modeling error
    \item use subdifferentials for non-smooth processes
    \item unified handling of heterogeneous observational data
    \end{itemize}
  \item PDE-constrained Optimization
    \begin{itemize}
    \item multi-objective
    \item robustness
    \item rich problem description
    \item fusing algorithmic steps (LNK and coupled DD fuse gradients with progress)
    \end{itemize}
  \item Exploring stability manifolds
    \begin{itemize}
    \item solving bordered linear and nonlinear systems
    \end{itemize}
  \item (nearly) time-periodic problems
    \begin{itemize}
    \item identifying cycles in ocean models, turbomachinery
    \end{itemize}
  \end{itemize}
\end{frame}

\begin{frame}
  \begin{itemize}
  \item discretizations with special properties
    \begin{itemize}
    \item more flops with less memory motion
    \item more accuracy with less memory motion or with more flops
    \item local conservation for FP/memory error detection
    \item high order schemes with small vertex separators
    \item continuously differentiable when possible (e.g. TVD; enables adjoints, Newton)
    \item unified discretization frameworks enable multiphysics
    \end{itemize}
  \item circumventing order barriers
    \begin{itemize}
    \item high order downwind SSP time integrators
    \item positivity and maximum-principle preservation with high order accuracy
    \end{itemize}
  \item multi-rate time integration for stiff systems
  \item variational inequalities, contact
    \begin{itemize}
    \item awkwardly large spaces for inequalities
    \item detection
    \end{itemize}
  \item synchronization for particle in cell methods
  \end{itemize}
\end{frame}

\end{document}
