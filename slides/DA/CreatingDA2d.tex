\frame{
\frametitle{Creating a DMDA}

{\small \kb DMDACreate2d(comm, xbdy, ybdy, type, M, N, m, n, \\
\qquad\qquad\qquad  dof, s, lm[], ln[], DA *da)}
\begin{columns}\begin{column}{0.15\textwidth}\end{column}\begin{column}{0.85\textwidth}
  \begin{itemize}
  \item[{\kb xbdy,ybdy}:] Specifies periodicity or ghost cells
  \begin{itemize}
    \item {\kb DMDA\_BOUNDARY\_NONE}, {\kb DMDA\_BOUNDARY\_GHOSTED}, {\kb DMDA\_BOUNDARY\_MIRROR}, {\kb DMDA\_BOUNDARY\_PERIODIC}
  \end{itemize}
  \item[{\kb type}:] Specifies stencil
  \begin{itemize}
    \item {\kb DMDA\_STENCIL\_BOX} or {\kb DMDA\_STENCIL\_STAR}
  \end{itemize}
  \item[{\kb M,N}:] Number of grid points in x/y-direction
  \item[{\kb m,n}:] Number of processes in x/y-direction
  \item[{\kb dof}:] Degrees of freedom per node
  \item[{\kb s}:] The stencil width
  \item[{\kb lm,ln}:] Alternative array of local sizes
  \begin{itemize}
    \item Use {\kb PETSC\_NULL} for the default
  \end{itemize}
\end{itemize}
\end{column} \end{columns}
}
