\section{Installation}
\begin{frame}{Downloading}
\begin{itemize}
  \item \url{http://mcs.anl.gov/petsc}, download tarball
  \item We will use Mecurial in this tutorial:
  \begin{itemize}
    \item \url{http://mercurial.selenic.com}
    \item Debian/Ubuntu: \shell{aptitude install mercurial}
    \item Fedora: \shell{yum install mercurial}
  \end{itemize}
  \item Get the PETSc release
  \begin{itemize}\footnotesize
    \item \shell{\small hg clone \mtab\bslash \\
     \scriptsize http://petsc.cs.iit.edu/petsc/petsc-releases/petsc-release-3.1}
    \item \shell{cd petsc-release-3.1}
    \item \shell{\scriptsize hg clone http://petsc.cs.iit.edu/petsc/ \\
          \qquad BuildSystem-releases/BuildSystem-release-3.1 \mtab\bslash \\
        \qquad config/BuildSystem}
    \item Get the latest bug fixes with \shell{hg pull --update}
  \end{itemize}
\end{itemize}
\end{frame}

\begin{frame}{Configuration}
\begin{block}{Basic configuration}
\begin{itemize}\footnotesize
  \item \shell{export PETSC\_DIR=\$PWD PETSC\_ARCH=mpich-gcc-dbg}
  \item \shell{./configure -{}-with-shared \mtab \bslash \\
  	\qquad\qquad -{}-with-blas-lapack-dir=/usr \mtab\bslash \\
  	\qquad\qquad -{}-download-\{mpich,ml,hypre\}}
  \item \shell{make all test}
\end{itemize}
\end{block}
\begin{itemize}
\item Other common options
  \begin{itemize}\footnotesize
  \item \item \code{-{}-with-mpi-dir=/path/to/mpi}
  \item \code{-{}-with-scalar-type=$<$real or complex$>$}
  \item \code{-{}-with-precision=$<$single,double,longdouble$>$}
  \item \code{-{}-with-64-bit-indices}
  \item \code{-{}-download-\{umfpack,mumps,scalapack,blacs,parmetis\}}
  \end{itemize}
\item reconfigure at any time with \\
  {\footnotesize \shell{mpich-gcc-dbg/conf/reconfigure-mpich-gcc-dbg.py \mtab\bslash\\
      \qquad\qquad -{}-new-options}}
\end{itemize}
\end{frame}

\frame{
\frametitle{Automatic Downloads}

\begin{itemize}
  \item Starting in 2.2.1, some packages can be automatically
  \begin{itemize}
    \item Downloaded
    \item Configured and Built (in {\kb \$PETSC\_DIR/externalpackages})
    \item Installed with PETSc
  \end{itemize}

  \item Currently works for
  \begin{itemize}
    \item petsc4py
    \item PETSc documentation utilities (Sowing, lgrind, c2html)
    \item BLAS, LAPACK, BLACS, ScaLAPACK, PLAPACK
    \item MPICH, MPE, Open MPI
    \item ParMetis, Chaco, Jostle, Party, Scotch, Zoltan
    \item MUMPS, Spooles, SuperLU, SuperLU\_Dist, UMFPack, pARMS
    \item PaStiX, BLOPEX, FFTW, SPRNG
    \item Prometheus, HYPRE, ML, SPAI
    \item Sundials
    \item Triangle, TetGen, FIAT, FFC, Generator
    \item HDF5, Boost
  \end{itemize}
\end{itemize}
\emph{Can also use \code{-{}-with-xxx=/path/to/your/install}}
}


\begin{frame}{An optimized build}
  \begin{itemize}
  \item \shell{\small mpich-gcc-dbg/conf/reconfigure-mpich-gcc-dbg.py \\
      PETSC\_ARCH=mpich-gcc-opt \\
      -{}-with-debugging=0 \&\& make PETSC\_ARCH=mpich-gcc-opt}
  \item What does \code{-{}-with-debugging=1} (default) do?
    \begin{itemize}
    \item Keeps debugging symbols (of course)
    \item Maintains a stack so that errors produce a full stack trace (even SEGV)
    \item Does lots of integrity checking of user input
    \item Places sentinels around allocated memory to detect memory errors
    \item Allocates related memory chunks separately (to help find memory bugs)
    \item Keeps track of and reports unused options
    \item Keeps track of and reports allocated memory that is not freed \\
      \quad \code{-malloc\_dump}
    \end{itemize}
  \end{itemize}
\end{frame}

\begin{frame}{Compiling an example}
\begin{itemize}
  \item \shell{hg clone \\
  {\footnotesize http://petsc.cs.iit.edu/petsc/tutorials/CSCS10TutorialCode}}
  \item \shell{cd CSCS2010TutorialCode}
  \item \shell{hg update -r\Rbasic}
  \item \shell{make}
  \item \shell{./pbratu -help | less}
\end{itemize}
\end{frame}
