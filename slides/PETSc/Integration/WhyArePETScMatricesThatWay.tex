\begin{frame}{Why Are PETSc Matrices That Way?}

\begin{itemize}
  \item No one data structure is appropriate for all problems
  \begin{itemize}
    \item Blocked and diagonal formats provide significant performance benefits
    \item PETSc has many formats and makes it easy to add new data structures
  \end{itemize}

  \item Assembly is difficult enough without worrying about partitioning
  \begin{itemize}
    \item PETSc provides parallel assembly routines
    \item Achieving high performance still requires making most operations local
    \item However, programs can be incrementally developed.
    \item {\kb MatPartitioning} and {\kb MatOrdering} can help
  \end{itemize}

  \item Matrix decomposition in contiguous chunks is simple
  \begin{itemize}
    \item Makes interoperation with other codes easier
    \item For other ordering, PETSc provides ``Application Orderings'' (AO)
  \end{itemize}
\end{itemize}

\end{frame}
