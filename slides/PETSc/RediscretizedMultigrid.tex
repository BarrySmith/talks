\begin{frame}{Rediscretized Multigrid using \texttt{DM}}
  \begin{itemize}
{\scriptsize
  \item \texttt{DM} manages problem data beyond purely algebraic objects
    \begin{itemize}
    \item structured, redundant, and (less mature) unstructured implementations in PETSc
    \item third-party implementations
    \end{itemize}
  \item \texttt{DMCoarsen(dmfine,coarse\_comm,\&coarsedm)} to create ``geometric'' coarse level
    \begin{itemize}
    \item Also \texttt{DMRefine()} for grid sequencing and convenience
    \item \texttt{DMCoarsenHookAdd()} for external clients to move resolution-dependent data for rediscretization and FAS
    \end{itemize}
  \item \texttt{DMCreateInterpolation(dmcoarse,dmfine,\&Interp,\&Rscale)}
    \begin{itemize}
    \item Usually uses geometric information, can be operator-dependent
    \item Can be improved subsequently, e.g. using energy-minimization from AMG
    \end{itemize}
  \item \texttt{DMCreateDecomposition(dm,\&nsplits,\&splitnames,\&splits,\&dms)}
    \begin{itemize}
    \item New/tentative API to expose split information to preconditioner and nonlinear solvers
    \item Can have multiple named decompositions
    \end{itemize}
  \item Resolution-dependent solver-specific callbacks use attribute caching on \texttt{DM}.
    \begin{itemize}
    \item Managed by solvers, not visible to users unless they need exotic things (e.g. custom homogenization, reduced models)
    %\item This implementation aspect is subject to change, but should not affect user interface.
    \end{itemize}
}
  \end{itemize}
\end{frame}
